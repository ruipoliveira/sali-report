%%%%%%%%%%%%%%%%%%%%%%%%%%%%%%%%%%%%%%%%%%%%%%%%%%%%%%%%%%%%%%%%%%%%%%%%%%%%%%%%%%%%%%%%%%%%%%%%%%%
\chapter{Conclusões e trabalho futuro}


\section{Conclusões}



O objetivo deste trabalho era desenhar e desenvolver um sistema de informação que permitisse a recolha de dados provenientes de um sistema de sensores para monitorizar e controlar o cultivo da Salicórnia. O trabalho prático desta dissertação foi elaborado tendo por base este objetivo geral e pode-se afirmar que este foi cumprido com sucesso. Este sistema disponibiliza uma plataforma web que permite aos utilizadores consultar os dados obtidos pelos sensores e atuar remotamente permitindo melhorar as condições de cultivo. Para além disso, é disponibilizada uma \ac{API} que permite o acesso a serviços do sistema, possibilitando a criação de novas aplicações. Para simular este cenário, foi criado um protótipo de \textit{hardware} para testar o quadro pretendido. Adicionalmente, foi criado um sistema de vídeo-vigilância para incorporar nas quintas onde se faz a produção desta planta. Todas estas funcionalidades vão de encontro aos objetivos específicos apresentados na secção \ref{objectivos}, à exceção da incorporação do sistema de vídeo-vigilância com o algoritmo de deteção de intrusos. Contudo, este algoritmo foi apresentado e testado para alguns cenários. 

Toda a modelação do sistema vai de encontro aos requisitos inicialmente especificados pelo cliente, bem como aos definidos durante o desenvolvimento deste trabalho. Desta forma, é permitido que o sistema seja genérico e passível de ser aplicado a qualquer cenário, seguindo a arquitetura definida. Adicionalmente aos objetivos desta dissertação, planeou-se a arquitetura e criou-se um mockup de uma aplicação mobile, estando esta prevista pelos requisitos do cliente. 




\section{Problemas encontrados}


Durante o desenvolvimento e implementação deste sistema surgiram alguns problemas, tanto pontuais e de correção simples, como
problemas estruturais, que levaram a algumas mudanças. Alguns dos problemas estruturais estão relacionados com o modelo de dados, em que foram adicionados novos campos às tabelas existentes, quer para o suporte de novas funcionalidades, quer para aumentar o comportamento dinâmico do sistema.

Tal como dito anteriormente, não foi possível incorporar o sistema de vídeo-vigilância com o algoritmo de deteção de intrusos, sendo que se pretendia utilizar a \ac{API} do Youtube. Esta utilização não foi possível devido à dificuldade da API e à falta de documentação. 


Um outro obstáculo na realização deste trabalho, foi o facto de o projeto não ser financiado por parte do cliente, impossibilitando assim, a compra de um sensor de salinidade, sendo este um dos parâmetros mais importante de monitorizar no controlo do cultivo da Salicórnia. Para além disso, pretendia-se adquirir uma câmara de maior qualidade e outra térmica, permitindo a deteção de intrusos durante a noite. 




%projeto sem financiamento por nao foram utilizados sensores de salinidade: 


%justificar o falta fazer se é estável pode ser usado como porto de partida para 

%O que podia ser feito: testes de usabilidade, medir tempos de resposta do web site; \\



\section{Trabalho futuro}



Como trabalho futuro, propõe-se realizar alguns testes de usabilidade à aplicação web desenvolvida, permitindo verificar o grau de facilidade/dificuldade deste \textit{software}. 





registo automatico dos módulo através de QR por exemplo\\













 
