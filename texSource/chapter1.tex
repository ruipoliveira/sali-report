%%%%%%%%%%%%%%%%%%%%%%%%%%%%%%%%%%%%%%%%%%%%%%%%%%%%%%%%%%%%%%%%%%%%%%%%%%%%%%%%%%%%%%%%%%%%%%%%%%%
\chapter{Introdução}




\begin{figure}[!htb]
\centering
\includegraphics{uaLogoNew.pdf}
\caption{Salicornia proveniente da ria de Aveiro}
\label{Rotulo}
\end{figure}








http://eusougourmet.blogspot.pt/2011/09/compre-o-que-e-nosso-salicornia.html







\section{Objetivos}

Este trabalho tem como objetivo o desenvolvimento

\begin{itemize}
    \item Criação de uma plataforma web que permita: 

    \begin{itemize}
        \item Disponibilizar a leitura dos mais diversos sensores de sensores (temperatura, salinidade...)
        
        \item Permitir gerar alarmes de inundação, sendo este enviados via SMS ou email para o cliente. 
        
        \item Atuar remotamente para drenagem de água em excesso existente nas leiras
        
        \item Sistema de transmissão de vídeo disparada por eventos gerados pelos sensores
        
        
    \end{itemize}
    
    \item Criação de uma aplicação móvel que permita receber alarmismos de situações anómalas. 
\end{itemize}


\section{Organização do documento}




No Capítulo 2 apresenta-se 



o projeto CAMBADA e identifica-se os pontos chave tanto
do software como do hardware. No Capítulo 3 


No Capítulo 4 é.... 

Para finalizar, no Capıtulo 5 apresentam-se conclusões sobre o trabalho desenvolvido e eventuais melhorias para o futuro.









