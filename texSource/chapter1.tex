%%%%%%%%%%%%%%%%%%%%%%%%%%%%%%%%%%%%%%%%%%%%%%%%%%%%%%%%%%%%%%%%%%%%%%%%%%%%%%%%%%%%%%%%%%%%%%%%%%%
\chapter{Introdução}




\begin{figure}[!htb]
\centering
\includegraphics{uaLogoNew.pdf}
\caption{Salicornia proveniente da ria de Aveiro}
\label{Rotulo}
\end{figure}



\section{Motivação}




http://eusougourmet.blogspot.pt/2011/09/compre-o-que-e-nosso-salicornia.html


\ac{HTML}



* O gênero salicornia \ac{HTML} inclui cerca de 117 espécies, sendo Salicomia herbacea, Salicornin bigelovii, Salicornia europea, Snlicornia prostata, Salicorn ia mmosissima e Salicornia verginica aquelas com maior ocorrência. \cite{overviewsal}  \cite{Saini2014}

A que serve de mote a esta dissertação ...




Os recursos naturais, nomeadamente, plantas, animais e minerais, são utilizados desde a antiguidade pelo ser humano, não apenas como fonte de alimentos mas também para o tratamento de diversas doenças []. Muitas das espécies que nascem em todo o mundo inicialmente são consideradas pragas, contudo e após alguns estudos intensivos à espécie são descobertas verdadeiras pérolas. Um exemplo disso é a salicornia.

A salicornia é a planta que iremos dar destaque durante este projeto. Esta planta é por vezes utilizada como substituta do sal marinho[] e utilizada para os mais diversos fins. Iremos abordar alguns deles mais à frente. 

A salicornia nasce e cresce naturalmente ao longo dos estuários e sapais (salinas) costeiras do Mediterrâneo[]. 





Esta é uma planta suculenta adaptada a ambientes salinos (halófita) que se desenvolve maioritáriamente em ambientes aquários com elevado teor de sal.[] 






Existem mais de  de  as mais comuns são: 






Existem cerca de uma centena de espécies do género Salicornia L.[], as mais comum encontram-se destacadas de seguida: 

Salicornia virginica: é uma planta com flor e pode ser encontrada na região mediterrânica
Salicornia europea: resce em várias zonas de entre-marés salinas 
Salicornia maritima: 
Salicornia bigelovii: 
Salicornia perennis: 
Salicornia ramosissima: 






A evolução tecnológica é algo que sempre esteve presente na vida do ser humano desde os seus primórdios até aos dias atuais, sendo que se tem verificado um aumento desta relação com o humano e principalmente com o ritmo da própria evolução.  As tecnologias, de uma maneira geral, são todas as invenções produzidas pelo homem, para aumentar a sua atividade no planeta e simplificar o modo de vida que quem o habita [1]. O conceito de “Internet das coisas” (do inglês “Internet of Things”, IoT) é fruto desta evolução tecnológica, já que permite a ligação dos mais diversos dispositivos eletrónicos à Internet. 









\section{Objetivos}

Este trabalho tem como objetivo o desenvolvimento

\begin{itemize}
    \item Criação de uma plataforma web que permita: 

    \begin{itemize}
        \item Disponibilizar a leitura dos mais diversos sensores de sensores (temperatura, salinidade...)
        
        \item Permitir gerar alarmes de inundação, sendo este enviados via SMS ou email para o cliente. 
        
        \item Atuar remotamente para drenagem de água em excesso existente nas leiras
        
        \item Sistema de transmissão de vídeo disparada por eventos gerados pelos sensores
        
        
    \end{itemize}
    
    \item Criação de uma aplicação móvel que permita receber alarmismos de situações anómalas. 
\end{itemize}


\section{Organização do documento}


A presente dissertação está dividida em 7 capítulos: ...

O primeiro capítulo descreve e enfatiza a importânci

De seguida, no Estado de Arte, é




No Capítulo 2 apresenta-se 



o projeto CAMBADA e identifica-se os pontos chave tanto
do software como do hardware. No Capítulo 3 


No Capítulo 4 é.... 

Para finalizar, no Capıtulo 5 apresentam-se conclusões sobre o trabalho desenvolvido e eventuais melhorias para o futuro.









