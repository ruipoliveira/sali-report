
\chapter{Conceito de IoT no cultivo da Salicórnia}

A palavra salicórnia deriva do latim tardio \textit{sal}, que significa sal, e \textit{cornus} que significa corno. Etimologicamente a palavra salicórnia significa cornos salgados\cite{chambers}. A espécie de salicórnia que que servirá de mote à elaboração desta dissertação é a única existente em Portugal designada por \sr \textit{J. Woods (S. ramosissima)}\cite{JoaoSilva}, uma espécie do género \textit{Salicornia L.}, pertencente à família das beterrabas denominada de \textit{Chenopodiaceae}\cite{chenopodiaceae}.

Nesta secção será apresentada a \sr que impulsionará toda esta dissertação. Serão descritas as principais características desta planta, principais propriedades e as diferentes aplicações alimentais existentes no mercado. Para além disso, será feita uma pequena introdução ao conceito de \ac{IoT} e respetiva importância no contexto deste projeto.


\section{Características da planta}


A salicórnia é uma espécie halófita, ou seja adaptada a viver em ambientes com elevado teor salino\cite{ferri}, sendo uma das mais evoluídas da sua família. É uma planta anual de dimensão pequena, aparentemente sem folhas, ereta, os seus caules são carnudos e suculentos, simples e/ou extremamente ramificados, segmentados por articulações\cite{Silva2000}, geralmente com menos de 30 cm de altura\cite{overviewsal}.

A salicórnia tem uma coloração normalmente verde-escuro mas a sua ramagem torna-se  verde-amarelado ou mesmo vermelho-púrpura no outono\cite{Silva2000}. A figura \ref{primoutono} ilustra a respetiva coloração na primavera e no outono. Na Inglaterra, a salicórnia é conhecida como \textit{purple glasswort}, podendo este nome estar na origem desta pigmentação caraterística\cite{Davy2001}. Em Portugal e Espanha é conhecida vulgarmente como erva-salada, sal verde e/ou espargos do mar\cite{RaquelPinto}. 



\begin{figure}[h]
	\centering
	\begin{minipage}[b]{0.49\textwidth}
		\includegraphics[width=\textwidth]{img/cap2-sali/Salicornia04.JPG}
	\end{minipage}
	\hfill
	\begin{minipage}[b]{0.49\textwidth}
		\includegraphics[width=\textwidth]{img/cap2-sali/sal-outono.png}
	\end{minipage}
	\caption{\sr: na primavera e no outono respetivamente à esquerda e à direita (Fotografia por José M. G. Pereira)}
	\label{primoutono}
\end{figure}





A \sr desenvolve-se preferencialmente no litoral costeiro, em pântanos e sapais salgados ou em margens de salinas temporariamente alagadas. Encontra-se distribuída maioritariamente na parte oeste da Europa e a oeste da região do Mediterrâneo, sendo uma das espécies mais abundante\cite{Figueroa1987}. Pode ser encontrada em todo o litoral da Península Ibérica, embora com menos frequência no Minho\cite{Silva2000}. Em Portugal, é encontrada ao longo da costa, mais frequentemente nas margens dos canais da Ria de Aveiro e Ria Formosa, no Algarve\cite{RaquelPinto}. 

Esta planta é uma das mais estudadas a nível mundial\cite{Figueroa1987}, possuindo um ciclo de vida anual bem definido, com gerações discretas e as suas sementes são hermafroditas\cite{Silva2007}. A salicórnia cresce habitualmente entre março, início da sementeira (A da figura \ref{ciclodevida}) com respetivo crescimento (B da figura \ref{ciclodevida}) e novembro fechando assim o ciclo com a produção de sementes (E da figura \ref{ciclodevida}). Entre maio  e agosto decorre a colheita da planta\cite{RaquelPinto} (C da figura \ref{ciclodevida}) utilizada para os mais diversos fins. A floração ocorre fundamentalmente no mês de outubro\cite{Figueroa1987} (D da figura \ref{ciclodevida}). 



	
\begin{figure}[!htb]
	\centering
	\includegraphics[width=\linewidth]{img/cap2-sali/ciclo/ciclodevida.pdf}
	\caption{Ciclo de vida da \sr (Fotografia por Helena Silva)}
	\label{ciclodevida}
\end{figure}


\newpage

\section{Importância da planta}


Uma das características que tornam o género \textit{Salicornia L} uma planta tão popular são as suas elevadas propriedades nutricionais, nomeadamente a nível de minerais e vitaminas antioxidantes, como vitamina C e $\beta$-caroteno. A salicornia é também uma fonte de proteínas e possui um alto teor total de lípidos e ómega-3\cite{Ventura2011}.


Desde a descoberta da salicórnia que esta é usada a nível culinário mas também no tratamento e prevenção de algumas doenças. Seguidamente irei aprofundar cada uma dessas aplicações esclarecendo a sua relevância. 



\subsection{Aplicações alimentares}


Espécies do género \textit{Salicornia L.} estão incluídas na alimentação humana, desde a antiguidade, sendo normalmente consumida crua, cozinhada ou seca, podendo ser triturada. Quando crua é usada como acompanhamento das mais diversas refeições enquanto que seca ou triturada é usada como especiaria, podendo ser utilizada como tempero na confeção de peixes, marisco ou carnes. O sal verde é um grande substituto do sal comum, pois é rico em substâncias depurativas e diuréticas. Os seus caules carnudos são bastante requisitados para cozinhas \textit{gourmet}, não só pelo seu sabor salgado, mas também pelo seu elevado valor nutricional\cite{Filomena2009}.


 

%especifiaria, conhecida como sal verde, podendo ser utilizado maioritariamente para tempero 

%A Salicórnia seca e triturada, transforma-se numa especiaria – Sal Verde – podendo ser utilizada como tempero. O Sal Verde é mais vantajoso em relação ao sal comum, pois é rico em substâncias depurativas e diuréticas (Raposo et al., 2009).

%A Salicórnia pode ser consumida crua ou cozinhada. Crua, pode acompanhar saladas ou batatas. Em conserva de vinagre pode acrescentar uma nota ácida a diversos pratos. Cozida em água durante cerca de 10 minutos pode depois ser salteada em manteiga.


%Associada com frequência na confeção de peixe e marisco, conceituados chefs internacionais introduzem-na em pratos de carne, nomeadamente borrego.


\subsection{Aplicações medicinais}


A nível medicinal, existem inúmeros estudos que revelam as propriedades químicas que esta planta detém. Existem estudos que demonstram estas propriedades na prevenção e tratamento de algumas doenças, tais como, a hipertensão, cefaleias e escorbuto, diabetes, obesidade, cancro, entre outras\cite{Wang2012}.


\section{Condições ideais de cultivo da salicórnia}

O crescimento da \sr é influenciada pela salinidade do meio. Um estudo realizado por \textit{Silva et al.}\cite{Silva2007} comprova que esta planta halófita apresenta um crescimento ideal a salinidades baixas ou moderadas, em vez de salinidades elevadas, pelo que é considerada uma halófita não obrigatória.


%alterar bastante o texto... palha












%a \sr que impulsionará toda esta dissertação. Serão descritas as principais características desta planta, principais propriedades e as diferentes aplicações alimentais existentes no mercado. 



\section{Evolução tecnológica: o \acs{IoT}}


Antes de descrever a importância e o conceito de \ac{IoT}, é necessário entender as diferenças entre os termos Internet e\ac{WWW}, que 	são usados indistintamente pela sociedade. A Internet é a camada ou rede física composta por \textit{switches}, \textit{routers} e outros equipamentos\cite{Evans2011a}. A sua principal função é transportar informações de um ponto para outro de forma rápida, confiável e segura. Por outro lado, a Web pertence à camada de aplicações que opera sobre a Internet cuja função é oferecer uma interface que transforme as informações que fluem pela Internet em algo útil. Ao longo do tempo, a Web passou e continua a passar por várias etapas evolucionárias, identificadas como:

\begin{itemize}
	\item \textbf{Web 1.0 - passado}: esta primeira etapa foi inventada por Tim Berners Lee em 1989\cite{Getting}. Nesta fase surgiram os principais conceitos que conhecemos da Internet atual: Localizador Uniforme de Recursos (do inglês \ac{URL}), Linguagem de Marcação de Hipertexto (do inglês \ac{HTML}) e Protocolo de Transferência de Hipertexto (do inglês \ac{HTTP}). Ainda nesta primeira fase, mas mais tarde, em 1998 foi criado por Larry Page e Sergey Brin o Google que criou simplicidade nas pesquisas na Web\cite{Lovato2014}. 
	
	\item \textbf{Web 2.0 - presente}: a Web cresceu muito e muito rapidamente. A versão mais próxima da visão de Tim Berners Lee – colaborativa, usado como meio de interação, comunicação global e elevado compartilhamento de informação. 
	
	\item \textbf{Web 3.0 - futuro}: para o futuro prevê-se que os conteúdos \textit{online} possão vir a estar organizados de forma semântica, muito mais personalizados para cada utilizador, sites, aplicações inteligentes e/ou publicidade baseada nas pesquisas e nos comportamentos.
\end{itemize}

O aparecimento do \ac{IoT} foi extraordinariamente importante já que se trata da primeira evolução real da Internet, um salto que levará, no futuro, ao desenvolvimento de aplicações revolucionárias com potencial para melhorar significativamente a forma como a sociedade vive, aprende, trabalha e se diverte. O \ac{IoT} já transformou a Internet em algo sensorial, através da medição de diferentes características, como por exemplo a temperatura, a pressão, as vibrações, a iluminação, a humidade, o \textit{stress}, entre outras. 

A figura \ref{iotEvolution} representa a evolução da Internet em cinco fases. Inicialmente surge a conexão entre dois computadores que permite a criação de uma rede, posteriormente nasce o conceito de \ac{WWW} ligando um grande número de computadores entre si. Seguidamente, surgiu a Internet móvel que permitiu conectar dispositivos moveis à Internet, possibilitando a ligação da sociedade através das redes sociais.
Finalmente, a internet está a evoluir para o \ac{IoT}, permitindo ligar objetos do quotidiano ao sistema global de redes de computadores\cite{Our2013}.




\newpage

\begin{figure}[h]
	\centering
	\includegraphics[width=\linewidth]{esquemas/iot-diagram.pdf}
	\caption[Evolução da internet em cinco fases]{ Evolução da internet em cinco fases (Adaptado de \cite{Our2013})}
	\label{iotEvolution}
\end{figure}



Uma das principais vantagens do IoT é a sua ligação evidente a todos os objetos, o que por si só é uma ideia avassaladora. O volume de dados gerado por este tipo de ligação pode ser interpretado pelo modelo \ac{DIKW}\cite{Rowley2007}. Este modelo, também conhecido como pirâmide do conhecimento (Figura \ref{dikw1}), é uma hierarquia informacional utilizada especialmente nas áreas da ciência da informação e na gestão do conhecimento, onde cada camada acrescenta certos atributos sobre a anterior.


\begin{figure}[!htb]
	\centering
	\includegraphics[scale=0.3]{img/cap3-iot/dikw.png}
	\caption{Pirâmide do conhecimento: modelo DIKW}
	\label{dikw1}
\end{figure}



A ligação dos objetos à Internet acarreta benefícios visíveis à nossa sociedade, possibilitando um maior controlo e entendimento de como os sistemas interagem entre si e proporcionando uma melhor qualidade de vida a todos. Embora as vantagens se sobreponham às desvantagens não nos podemos esquecer que existem alguns problemas a nível segurança, privacidade, legislação e identidade.



\section{Considerações finais}


Como vimos nesta secção, as propriedades da Salicornia têm conduzido a um elevado interesse económico e ao aumento do seu desenvolvimento comercial. Existem enumeradas empresas a cultivar esta espécie para que possa ser comercializada para os mais diversos fins, sendo que grande parte já é exportada. 

Uma vez que a salicornia carece de um controlo minucioso de certos parâmetros que tem vindo a ser estudados, existe necessidade de criar um sistema que permite monitorizar todos esses parâmetros de forma a melhor as condições de produção desta espécie. 

O conceito de \ac{IoT} encontra-se visível neste contexto, uma vez que possibilitará a interligação de equipamentos eletrónicos que permitem melhorar a eficiência com que a produção desta espécie é realizada.  

 





