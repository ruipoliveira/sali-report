
\chapter{Salicórnia: caracterização, importância e cultivo}

A palavra salicórnia deriva do latim tardio \textit{sal}, que significa sal, e \textit{cornus} que significa corno. Etimologicamente a palavra salicórnia significa cornos salgados\cite{chambers}. A espécie de salicórnia que irei aborda no decorrer desta dissertação é a única existente em Portugal designada por \sr \textit{J. Woods (S. ramosissima)}\cite{JoaoSilva}, uma espécie do género \textit{Salicornia L.}, pertencente à família das beterrabas denominada de \textit{Chenopodiaceae} \cite{chenopodiaceae}.

Nesta secção será apresentada a \sr que impulsionará toda esta dissertação. Serão descritas as principais características desta planta, principais propriedades e as diferentes aplicações alimentais existentes no mercado. 

\section{Características da planta}


A salicórnia é uma espécie halófita, ou seja adaptada a viver em ambientes com elevado teor salino\cite{ferri}, sendo uma das mais evoluídas da sua família. É uma planta anual de dimensão pequena, aparentemente sem folhas, ereta, os seus caules são carnudos e suculentos, simples e/ou extremamente ramificados, segmentados por articulações\cite{Silva2000}, geralmente com menos de 30 cm de altura\cite{overviewsal}.

A salicórnia é normalmente verde-escuro mas a sua ramagem torna-se  verde-amarelado ou mesmo vermelho-púrpura no outono. A figura \ref{primoutono} ilustra a respetiva coloração na primavera e no outono. Na Inglaterra, a salicórnia denomina-se como sendo \textit{purple glasswort}, podendo ter origem nesta coloração caraterística. Em Portugal e Espanha é conhecida vulgarmente como erva-salada, sal verde e/ou espargos do mar[]. 


\begin{figure}[!htb]
	\centering
	\includegraphics{uaLogoNew.pdf}
	\caption{\sr: a) na primavera e b) no outono (Fotografia por José M. G. Pereira)}
	\label{primoutono}
\end{figure}


A \sr desenvolve-se preferencialmente no litoral costeiro, em pântanos e sapais salgados ou em margens de salinas temporariamente alagadas. Encontra-se distribuída maioritariamente na parte oeste da Europa e a oeste da região do Mediterrâneo. Pode ser encontrada em todo o litoral da Península Ibérica, embora com menos frequência no Minho\cite{Silva2000}. Em Portugal, é encontrada ao longo da costa, mais frequentemente nas margens dos canais da Ria de Aveiro e Ria Formosa, no Algarve[]. 

Esta planta possui um ciclo de vida anual, sendo que cresce habitualmente entre março (início da sementeira) e novembro fechando assim o ciclo com produção de sementes. Entre maio  e agosto decorre a colheita da planta\cite{RaquelPinto} utilizada para os mais diversos fins. A figura \ref{ciclodevida} encontra-se esquematizado o ciclo de vida da \sr. 

 \begin{figure}[!htb]
 	\centering
 	\includegraphics{uaLogoNew.pdf}
 	\caption{Ciclo de vida da \sr}
 	\label{ciclodevida}
 \end{figure}
 
 



\section{Importância da planta}


Uma das características que tornam esta planta tão popular são as suas elevadas propriedades nutricionais, nomeadamente a nível de minerais e vitaminas antioxidantes, como vitamina C e $\beta$-caroteno. (Ventura et al., 2011a)

Desde a descoberta da salicornai que esta é usada a nível culinário mas também no tratamento e prevenção de algumas doenças. Seguidamente iremos aprofundar cada uma dessas aplicações esclarecendo a sua relevância. 



\subsection{Aplicações alimentares}


A Salicórnia seca e triturada, transforma-se numa especiaria – Sal Verde – podendo ser utilizada como tempero. O Sal Verde é mais vantajoso em relação ao sal comum, pois é rico em substâncias depurativas e diuréticas (Raposo et al., 2009).

\subsection{Processo de secagem}




A Salicórnia pode ser consumida crua ou cozinhada. Crua,
pode acompanhar saladas ou batatas. Em conserva de vinagre pode acrescentar uma nota ácida a diversos pratos. Cozida em água durante cerca
de 10 minutos pode depois ser salteada em manteiga.

Os caules carnudos deste vegetal são bastante
requisitados para cozinhas gourmet, não só pelo seu sabor salgado, mas também pelo seu valor nutricional 




Associada com frequência na confeção de peixe e marisco, conceituados chefs internacionais introduzem-na em pratos de carne, nomeadamente borrego.



\subsection{Propriedades medicinais}



que engloba diversas
espécies com importantes aplicações na medicina tradicional, tais como tratamento de
hipertensão, cefaleias e escorbuto,[7] diabetes, obesidade, cancro,[8,9] entre outras.






\subsection{Condições ideias de cultivo da salicórnia}


O crescimento da S. ramosissima é influenciado pela salinidade do meio. Um estudo realizado por Silva et al.[14] comprova que esta halófita apresenta um crescimento ótimo a salinidades baixas ou moderadas, em vez de salinidades elevadas, pelo que é considerada uma halófita não obrigatória. \cite{Silva2007}




\subsection{Controlo}

\subsection{Monitorização}




Neste projeto, iremos considerar a produção de Salicornia ramosissima em cultura.



\subsection{Importância do controlo}


O crescimento da S.  é influenciado pela salinidade do meio. Um
estudo realizado por Silva et al.[14] comprova que esta halófita apresenta um crescimento
ótimo a salinidades baixas ou moderadas, em vez de salinidades elevadas, pelo que é
considerada uma halófita não obrigatória.


\subsection{O cliente - Horta dos Peixinhos, Lda}


Cultura e comercialização de Salicornia


A Horta dos Peixinhos, Lda tem NIF 513620699 e desenvolve a sua atividade com o CAE 03210 - Aquicultura em águas salgadas e salobras.











