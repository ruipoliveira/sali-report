\chapter{Implementação do \textit{trigger} \acs{SQL} }
\label{triggerSQLImpe}


Seguidamente encontra-se o \textit{script} SQL da implementação do \textit{stored procedure} e respetivo \textit{trigger}. Os dois últimos comandos permitem eliminar a \textit{stored procedure} e o \textit{trigger}, respetivamente. 

\begin{lstlisting}[
language=SQL,
showspaces=false,
basicstyle=\ttfamily,
numbers=left,
numberstyle=\tiny,
commentstyle=\color{gray},
basicstyle=\ttfamily\footnotesize
]
CREATE OR REPLACE FUNCTION alarm_occurred() returns trigger as $alarm$ 
DECLARE
varmax FLOAT;
varmin FLOAT;
BEGIN

varmax := (select max from saliapp_alarmssettings where id_sensor_id= new.id_sensor_id);
varmin := (select min from saliapp_alarmssettings where id_sensor_id= new.id_sensor_id);

IF (new.value >= varmax) THEN 
insert into saliapp_alarms (id_reading_id, checked, max_or_min) VALUES (new.id, 'f', 't');
return new;
END IF;
IF (new.value <= varmin) THEN 
insert into saliapp_alarms (id_reading_id, checked, max_or_min) VALUES (new.id, 'f', 'f');
return new;
END IF;

RETURN NULL;
END
$alarm$
LANGUAGE plpgsql;

create trigger trigger_alarm_occurred after insert on saliapp_reading
for each row execute procedure alarm_occurred(); 

DROP FUNCTION alarm_occurred(); 

DROP TRIGGER trigger_alarm_occurred ON saliapp_reading;


\end{lstlisting}
