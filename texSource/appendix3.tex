\chapter{Trigger SQL }


\begin{lstlisting}[
language=SQL,
showspaces=false,
basicstyle=\ttfamily,
numbers=left,
numberstyle=\tiny,
commentstyle=\color{gray}
]
CREATE OR REPLACE FUNCTION alarm_occurred()
RETURNS TRIGGER AS $alarm$

DECLARE
varmax FLOAT;
varmin FLOAT;
BEGIN

varmax := (SELECT max FROM saliapp_alarmssettings 
	WHERE id_sensor_id= new.id_sensor_id);
varmin := (SELECT min FROM saliapp_alarmssettings 
	WHERE id_sensor_id= new.id_sensor_id);

IF (new.value >= varmax) THEN 
	INSERT INTO 
	saliapp_alarms (id_reading_id, checked, max_or_min) 
	VALUES (new.id, 'f', 't');
	RETURN new;
END IF;
IF (new.value <= varmin) THEN 
	INSERT INTO 
	saliapp_alarms (id_reading_id, checked, max_or_min)
	VALUES (new.id, 'f', 'f');
	RETURN new;
END IF;

RETURN NULL;
END
$alarm$
LANGUAGE plpgsql;

CREATE TRIGGER trigger_alarm_occurred AFTER INSERT 
ON saliapp_reading
FOR EACH ROW EXECUTE PROCEDURE alarm_occurred()

DROP FUNCTION alarm_occurred()
DROP TRIGGER trigger_alarm_occurred ON saliapp_reading;

\end{lstlisting}
