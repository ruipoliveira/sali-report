

\chapter{Sistema de deteção de intrusos}


\section{Bibliotecas de processamento de imagem}


Usado biblioteca do opencv que permite detectar 
HOGDescriptor


Deteção de intrusos: 

http://www.pyimagesearch.com/2015/11/09/pedestrian-detection-opencv/



versão simplificada: http://www.pyimagesearch.com/2015/02/16/faster-non-maximum-suppression-python/



Servidor em falsk 


deploy 
	https://iotbytes.wordpress.com/python-flask-web-application-on-raspberry-pi-with-nginx-and-uwsgi/



Dataset: %http://www.robots.ox.ac.uk/ActiveVision/Research/Projects/2009bbenfold_headpose/project.html


é usado um detector HOG juntamente com um classificador linear SVM 





parametros do método detectMultiScale do opencv 

\begin{itemize}
	\item \texttt{img}: parâmetro obrigatório. 
	\item \texttt{hitThreshold}: parâmetro opcional. 
	\item \texttt{winStride}: parâmetro opcional. 
	\item \texttt{padding}: parâmetro opcional.  Os valores típicos para preenchimento incluem  (8, 8) ,  (16, 16) ,  (24, 24) , e  (32, 32) .
	
		
	\item \texttt{scale}: parâmetro opcional. 
	\item \texttt{finalThreshold}: parâmetro opcional. 
	\item \texttt{useMeanShiftGrouping}: parâmetro opcional. 
\end{itemize}





Neste contexto apenas foram utilizados os seguintes parâmetros winStride, scale, padding. 







\section{Flask}

Flask é considerada uma microframework web desenvolvida em Python e baseado nas bibliotecas WSGI Werkzeug e Jinja2. Escolhi esta microframework pois pretende-se que esta seja executada num microcontrolador com baixos recursos. Para além disso, considera-se ser de fácil aprendizagem relativamente ao Django (já abordado na capitulo XX) e com uma ótima documentação. 




\section{Servidor web NGNIX}

\section{Considerações finais}

