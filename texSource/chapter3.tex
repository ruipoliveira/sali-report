
\chapter{Soluções para Internet of Things}

\section{Da tecnologia ao IoT}


Antes de descrever a importância e o conceito de \ac{IoT}, é necessário entender as diferenças entre os termos Internet e \ac{WWW}, que são usados indistintamente pela sociedade. A Internet é a camada ou rede física composta por switches, routers e outros equipamentos [3]. A sua principal função é transportar informações de um ponto para outro de forma rápida, confiável e segura. Por outro lado, a Web pertence à camada de aplicações que opera sobre a Internet cuja função é oferecer uma interface que transforme as informações que fluem pela Internet em algo útil. Ao longo do tempo, a Web passou e continua a passar por várias etapas evolucionárias, identificadas como:

\begin{itemize}
	\item Web 1.0 – passado: esta primeira etapa foi inventada por Tim Berners Lee em 1989 [5]. Nesta fase surgiram os principais conceitos que conhecemos da Internet atual: Localizador Uniforme de Recursos (do inglês “Uniform Resource Locator”, URL), Linguagem de Marcação de Hipertexto (do inglês “HyperText Markup Language”, HTML) e Protocolo de Transferência de Hipertexto (do inglês “Hypertext Transfer Protocol”, HTTP). Ainda nesta primeira fase, mas mais tarde, em 1998 foi criado por Larry Page e Sergey Brin o Google que criou simplicidade nas pesquisas na Web [6]. 
	
	\item Web 2.0 – presente: a Web cresceu muito e muito rapidamente. A versão mais próxima da visão de Tim Berners Lee – colaborativa, usado como meio de interação, comunicação global e elevado compartilhamento de informação. 
	
	\item Web 3.0 – futuro: para o futuro prevê-se que os conteúdos online possão vir a estar organizados de forma semântica, muito mais personalizados para cada utilizador, sites, aplicações inteligentes e/ou publicidade baseada nas pesquisas e nos comportamentos.
\end{itemize}

O aparecimento do IoT foi extraordinariamente importante já que se trata da primeira evolução real da Internet, um salto que levará, no futuro, ao desenvolvimento de aplicações revolucionárias com potencial para melhorar significativamente a forma como a sociedade vive, aprende, trabalha e se diverte. O IoT já transformou a Internet em algo sensorial, através da medição de diferentes características, como por exemplo a temperatura, a pressão, as vibrações, a iluminação, a humidade, o stress, entre outras. 



\section{Comunicação}

\ac{IoT}

\ac{IoT}


\section{Produtos semelhantes}

